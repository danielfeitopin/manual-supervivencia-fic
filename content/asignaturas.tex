\chapter{Créditos y Asignaturas}

\section{Créditos}

La \acrshort{UDC} sigue el sistema \textbf{\acrfull{ECTS}}.

\begin{curiosityBox}
    El sistema \acrshort{ECTS} forma parte del Plan Bolonia y es utilizado por las universidades europeas para convalidar asignaturas y \textbf{cuantificar el trabajo relativo al estudiante} que trabaja bajo los grados del \acrfull{EEES}.
\end{curiosityBox}

Un \textbf{crédito} suele representar un \textbf{número determinado de horas} de trabajo del estudiante, incluyendo tanto las horas de clase como el tiempo dedicado al estudio y la realización de tareas y proyectos fuera del aula. La correspondencia entre asignaturas y créditos varía en función de la carga de trabajo que se espera del estudiante.

En general, los planes de estudio establecen un \textbf{número mínimo de créditos} que el estudiante debe completar para obtener su título.

\begin{infoBox}
    Los tres grados de la \acrshort{FIC} (\href{\linkPortalGEI}{\acrshort{GEI}}, \href{\linkPortalGCED}{\acrshort{GCED}} y \href{\linkPortalGIA}{\acrshort{GIA}}) requieren \textbf{240 créditos ECTS}.
\end{infoBox}

\subsection{Precio de los créditos}

El \textbf{precio por crédito aumenta} con cada matrícula en la misma asignatura.

\begin{curiosityBox}
    El precio por crédito de la \acrshort{UDC} es igual al del resto de universidades del Sistema Universitario de Galicia. Este se publica en el \acrfull{DOG} y puede ser consultado desde la \href{https://www.udc.es/es/matricula/}{página web de la UDC}.
\end{curiosityBox}

\subsection{Reconocimiento de créditos}

Es posible \textbf{obtener créditos} por actividades extracurriculares, cursos de idiomas, voluntariado o incluso actividades deportivas.

\begin{infoBox}
    Las actividades concretas que otorgan créditos pueden consultarse en la \href{https://www.udc.es/gl/normativa/academica/}{\textbf{normativa académica sobre reconocimiento de créditos}}.
\end{infoBox}

\subsection{Restricciones de créditos}

Los créditos también resultan de \textbf{gran importancia} para la realización de \textbf{actividades obligatorias} como las prácticas en empresas o el \acrfull{TFG}.

\subsection{Requisitos de permanencia}

Para poder \textbf{continuar los estudios} en la \acrshort{UDC} se requiere un \textbf{rendimiento mínimo}:

\begin{itemize}
    \item Alcanzar \textbf{12 créditos} \acrshort{ECTS} en el \textbf{primer año}.

    \item Alcanzar \textbf{60 créditos} \acrshort{ECTS} \textbf{cada tres cursos académicos}.
\end{itemize}

\begin{warningBox}
    En el caso de no cumplir con el mínimo de créditos tras el primer curso se puede solicitar la permanencia mediante una \textbf{instancia al decano} entregada en administración. En caso favorable, será obligatorio aprobar un mínimo de \textbf{36 créditos} \acrshort{ECTS} en el siguiente curso académico.
\end{warningBox}

\begin{infoBox}
    Las cantidades de créditos indicados corresponden a una matrícula a tiempo completo. En el caso de una matrícula a tiempo parcial correspondería la \textbf{mitad} de cada cantidad.
\end{infoBox}

\begin{importantBox}
    Todas las condiciones, procedimientos y excepciones están recogidas en la \href{https://www.udc.es/gl/normativa/academica/}{\textbf{normativa de gestión académica}}. Se recomienda encarecidamente su consulta.
\end{importantBox}


\section{Asignaturas}

\subsection{Guías docentes}

Las \href{https://guiadocente.udc.es/guia_docent/index.php?centre=614&ensenyament=614G01&consulta=assignatures&idioma=cast}{\textbf{guías docentes}} recogen la información sobre las asignaturas de los grados y másteres. Esta información incluye al profesorado, los contenidos, la metodología de evaluación y una breve descripción de la asignatura.  

\begin{curiosityBox}
    Las guías docentes de un curso son aprobadas a finales del curso académico anterior, por lo que pueden ser consultadas antes de realizar la matrícula.
\end{curiosityBox}

\subsection{Resultados}

Para cada asignatura de los últimos cursos anteriores es posible consultar el número total de \textbf{aprobados} (~\textcolor{resultPassed}{\fcolorbox{black}{resultPassed}{\rule{5pt}{5pt}}}~), \textbf{suspensos} (~\textcolor{resultFailed}{\fcolorbox{black}{resultFailed}{\rule{5pt}{5pt}}}~) y \textbf{no presentados} (~\textcolor{resultNotPresented}{\fcolorbox{black}{resultNotPresented}{\rule{5pt}{5pt}}}~) en el \href{\linkPortalEstudos}{portal de estudios}.

\FloatBarrier
\begin{figure}[htp]
    \centering
    \frame{\includegraphics[width=0.8\linewidth]{figures/asignaturas/resultados.png}}
\end{figure}
\FloatBarrier

\subsection{Tutorías} 

El profesorado cuenta con horarios dedicados a atender a los alumnos para resolver dudas que no se han resuelto en clase o para recibir atención más personalizada.  

Los \textbf{horarios y lugares de tutorías} de los profesores de la \acrshort{FIC} pueden ser consultados en la \href{https://www.udc.es/es/centros_departamentos_servizos/centros/titorias/?codigo=614}{página de tutorías de la \acrshort{UDC}} o en \href{https://espazos.udc.es/centers/614/tutorials}{Espazos \acrshort{UDC}}.

\begin{infoBox}
    Algunos profesores solicitan concertar una cita previa para sus tutorías, otros prefieren que, en caso de dudas puntuales, se les envíe dicha duda en un mensaje por Teams o correo electrónico. Es recomendable preguntar al profesorado por sus preferencias.     
\end{infoBox}
 
\subsection{Evaluación docente}

Al finalizar cada cuatrimestre, se podrán realizar unas \textbf{encuestas anónimas} para \textbf{valorar al profesorado} de cara a mejorar las clases y las asignaturas. Existe una encuesta por cada docente de las asignaturas en las que se está matriculado y debe llegarse a un \textbf{mínimo de encuestas} para que se tengan en cuenta. Estas encuestas tienen gran repercusión en la \textbf{reputación} del profesorado, por lo que es importante que se realicen de manera objetiva y sincera. 

\subsection{Evaluación por compensación}

Si a un estudiante solo le falta una asignatura para obtener el título y cumple unos determinados requisitos, puede solicitar su compensación y completar el estudio.

\begin{infoBox}
    No se pueden compensar ni las prácticas en empresa ni el \acrshort{TFG}.
\end{infoBox}
